\documentclass[]{book}
\usepackage{lmodern}
\usepackage{amssymb,amsmath}
\usepackage{ifxetex,ifluatex}
\usepackage{fixltx2e} % provides \textsubscript
\ifnum 0\ifxetex 1\fi\ifluatex 1\fi=0 % if pdftex
  \usepackage[T1]{fontenc}
  \usepackage[utf8]{inputenc}
\else % if luatex or xelatex
  \ifxetex
    \usepackage{mathspec}
  \else
    \usepackage{fontspec}
  \fi
  \defaultfontfeatures{Ligatures=TeX,Scale=MatchLowercase}
\fi
% use upquote if available, for straight quotes in verbatim environments
\IfFileExists{upquote.sty}{\usepackage{upquote}}{}
% use microtype if available
\IfFileExists{microtype.sty}{%
\usepackage{microtype}
\UseMicrotypeSet[protrusion]{basicmath} % disable protrusion for tt fonts
}{}
\usepackage[margin=1in]{geometry}
\usepackage{hyperref}
\hypersetup{unicode=true,
            pdftitle={Algèbre linéaire et géométrie vectorielle},
            pdfauthor={Marc-André Désautels},
            pdfborder={0 0 0},
            breaklinks=true}
\urlstyle{same}  % don't use monospace font for urls
\usepackage{natbib}
\bibliographystyle{apalike}
\usepackage{longtable,booktabs}
\usepackage{graphicx,grffile}
\makeatletter
\def\maxwidth{\ifdim\Gin@nat@width>\linewidth\linewidth\else\Gin@nat@width\fi}
\def\maxheight{\ifdim\Gin@nat@height>\textheight\textheight\else\Gin@nat@height\fi}
\makeatother
% Scale images if necessary, so that they will not overflow the page
% margins by default, and it is still possible to overwrite the defaults
% using explicit options in \includegraphics[width, height, ...]{}
\setkeys{Gin}{width=\maxwidth,height=\maxheight,keepaspectratio}
\IfFileExists{parskip.sty}{%
\usepackage{parskip}
}{% else
\setlength{\parindent}{0pt}
\setlength{\parskip}{6pt plus 2pt minus 1pt}
}
\setlength{\emergencystretch}{3em}  % prevent overfull lines
\providecommand{\tightlist}{%
  \setlength{\itemsep}{0pt}\setlength{\parskip}{0pt}}
\setcounter{secnumdepth}{5}
% Redefines (sub)paragraphs to behave more like sections
\ifx\paragraph\undefined\else
\let\oldparagraph\paragraph
\renewcommand{\paragraph}[1]{\oldparagraph{#1}\mbox{}}
\fi
\ifx\subparagraph\undefined\else
\let\oldsubparagraph\subparagraph
\renewcommand{\subparagraph}[1]{\oldsubparagraph{#1}\mbox{}}
\fi

%%% Use protect on footnotes to avoid problems with footnotes in titles
\let\rmarkdownfootnote\footnote%
\def\footnote{\protect\rmarkdownfootnote}

%%% Change title format to be more compact
\usepackage{titling}

% Create subtitle command for use in maketitle
\providecommand{\subtitle}[1]{
  \posttitle{
    \begin{center}\large#1\end{center}
    }
}

\setlength{\droptitle}{-2em}

  \title{Algèbre linéaire et géométrie vectorielle}
    \pretitle{\vspace{\droptitle}\centering\huge}
  \posttitle{\par}
    \author{Marc-André Désautels}
    \preauthor{\centering\large\emph}
  \postauthor{\par}
      \predate{\centering\large\emph}
  \postdate{\par}
    \date{2019-05-16}

\usepackage{booktabs}
\usepackage{amsthm}
\usepackage{multido}
\usepackage[french]{babel}

\hypersetup{colorlinks=true, urlcolor=blue}

\renewcommand{\chaptername}{Chapitre}
\renewcommand{\contentsname}{Table des Matières}
\renewcommand{\partname}{Partie}
\renewcommand\bibname{Bibliographie}

\begin{document}
\maketitle

{
\setcounter{tocdepth}{2}
\tableofcontents
}
\hypertarget{introduction}{%
\chapter*{Introduction}\label{introduction}}
\addcontentsline{toc}{chapter}{Introduction}

\hypertarget{a-propos-de-ce-document}{%
\section*{À propos de ce document}\label{a-propos-de-ce-document}}
\addcontentsline{toc}{section}{À propos de ce document}

\hypertarget{remerciements}{%
\subsection*{Remerciements}\label{remerciements}}
\addcontentsline{toc}{subsection}{Remerciements}

Ce document est généré par l'excellente extension \href{https://bookdown.org/}{bookdown} de \href{https://yihui.name/}{Yihui Xie}.

\hypertarget{license}{%
\subsection*{License}\label{license}}
\addcontentsline{toc}{subsection}{License}

Ce document est mis à disposition selon les termes de la \href{http://creativecommons.org/licenses/by-nc-sa/4.0/}{Licence Creative Commons Attribution - Pas d'Utilisation Commerciale - Partage dans les Mêmes Conditions 4.0 International}.

\begin{figure}
\centering
\includegraphics{resources/icons/license_cc.png}
\caption{Licence Creative Commons}
\end{figure}

\hypertarget{part-lalgebre-matricielle}{%
\part{L'algèbre matricielle}\label{part-lalgebre-matricielle}}

\hypertarget{sel}{%
\chapter{Les systèmes d'équations linéaires}\label{sel}}

\hypertarget{sec:intro_sel}{%
\section{Une introduction aux systèmes d'équations linéaires}\label{sec:intro_sel}}

Dans cette section, nous introduisons les notions d'équation linéaire et de système d'équations linéaires. Nous introduisons la façon de résoudre de petits systèmes d'équations linéaires. En pratique, les systèmes d'équations linéaires sont résolus grâce aux ordinateurs. Ces systèmes contiennent habituellement des centaines, des milliers (même des millions) d'équations et d'inconnues.

Intuitivement, une équation linéaire est une équation où toutes les variables sont affectées de l'exposant \(1\) et ne sont pas multipliées entre elles.

\bibliography{book.bib,packages.bib}


\end{document}
